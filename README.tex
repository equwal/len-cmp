% Created 2019-09-14 Sat 13:29
% Intended LaTeX compiler: pdflatex
\documentclass[11pt]{article}
\usepackage[utf8]{inputenc}
\usepackage[T1]{fontenc}
\usepackage{graphicx}
\usepackage{grffile}
\usepackage{longtable}
\usepackage{wrapfig}
\usepackage{rotating}
\usepackage[normalem]{ulem}
\usepackage{amsmath}
\usepackage{textcomp}
\usepackage{amssymb}
\usepackage{capt-of}
\usepackage{hyperref}
\author{Spenser Truex}
\date{\today}
\title{Length Comparison}
\hypersetup{
 pdfauthor={Spenser Truex},
 pdftitle={Length Comparison},
 pdfkeywords={},
 pdfsubject={},
 pdfcreator={Emacs 27.0.50 (Org mode 9.2.4)}, 
 pdflang={English}}
\begin{document}

\maketitle
\tableofcontents

\section{Purpose}
\label{sec:org3bdca94}
Provide \emph{short-circuit} length comparisons.
\section{Usage}
\label{sec:org1c4a8ca}
The only external function is `len`.
\begin{verbatim}
(len <integer or list> <list> &optional (<predicate> #'=))
(len 3 '(1 2 3)) ;=> T
(len '(a b c) '(1 2 3)) ;=> T
(len 3 '(1 2 3) #'/=) ;=> NIL
\end{verbatim}
\section{Complexity}
\label{sec:orgeab79b2}
Comparing the length of some lists with lengths n\textsubscript{i} in \textbf{N} requires going
across all of their elements.
\begin{verbatim}
\cal(\sum_{n=0}^{n-1}n_i)
\end{verbatim}

With a short-circuit length comparsion, the worst case scenario is the same, with the best case O(1) and average case O(min(N)).
\end{document}
